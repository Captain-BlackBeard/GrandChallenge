\documentclass{sig-alternate}

\usepackage{color}
\usepackage{subfigure}
\usepackage{booktabs}
\usepackage{ifpdf}
\usepackage{graphicx}
\usepackage[table]{xcolor}
\usepackage{microtype}

\clubpenalty=10000
\widowpenalty=10000

\setlength{\paperheight}{11in}
\setlength{\paperwidth}{8.5in}
\usepackage[
pass,% keep layout unchanged 
% showframe,% show the layout
]{geometry}

\usepackage{listings}
\lstset{	basicstyle=\small, 
			numbers=left, 
			xleftmargin=2em,
			frame=none,
			framexleftmargin=1.5em,
			numberstyle=\tiny, 
			tabsize=3, 
			language={Java}, 
			showstringspaces=false,
			commentstyle=\color[cmyk]{0.3,0.3,0.3,0.3},
			stringstyle=\color[cmyk]{0.92,0,0.95,0.40},
			keywordstyle=\color{blue}\bfseries,
			showspaces=false, 
			showtabs=false}


\newcommand{\TITLE}{The DEBS 2016 Grand Challenge}
\newcommand{\KEYWORDS}{event processing, streaming, utilities, geo-spatial}
\newcommand{\VG}{Vincenzo Gulisano}
\newcommand{\VGEMAIL}{{\normalsize vincenzo.gulisano@chalmers.se}}
\newcommand{\ZJ}{Zbigniew Jerzak}
\newcommand{\ZJEMAIL}{{\normalsize zbigniew.jerzak@sap.com}}
\newcommand{\SV}{Spyros Voulgaris}
\newcommand{\SVEMAIL}{{\normalsize spyros@cs.vu.nl}}
\newcommand{\HZ}{Holger Ziekow}
\newcommand{\HZEMAIL}{{\normalsize zie@hs-furtwangen.de}}


\newcommand{\VGADDR}[1]{	\affaddr{Chalmers University of Technology}\\%
							\affaddr{H\"{o}rsalsv\"{a}gen 11}\\%	
							\affaddr{41296 Gothenburg, Sweden}\\%
							\email{#1}}							
\newcommand{\ZJADDR}[1]{	\affaddr{SAP SE}\\%
							\affaddr{M\"{u}nzstra{\ss}e 15}\\%
							\affaddr{10178 Berlin, Germany}\\%
							\email{#1}}
\newcommand{\SVADDR}[1]{	\affaddr{Vrije Universiteit Amsterdam}\\%
							\affaddr{De Boelelaan 1081A}\\%	
							\affaddr{1081HV Amsterdam, The Netherlands}\\%
							\email{#1}}							
\newcommand{\HZADDR}[1]{	\affaddr{Hochschule Furtwangen}\\%
							\affaddr{Robert-Gerwig-Platz 1}\\%	
							\affaddr{78120 Furtwangen, Germany}\\%
							\email{#1}}							

\ifpdf
	\usepackage[pdftex,%
					plainpages=false,%
					pdfpagelabels=false,
					bookmarksnumbered,%
					colorlinks=true,%
					linkcolor=blue,%
					citecolor=blue,%
					unicode=true]{hyperref} 
	\usepackage{pdfpages}
	\usepackage[all]{hypcap}
	\hypersetup{%
		pdftitle={\TITLE},
		pdfauthor={\VG, \ZJ, \SV, \HZ},
		pdfsubject={\TITLE},
		pdfkeywords={\KEYWORDS},
		bookmarksopen=false,
		unicode=true,
		colorlinks=true,
		hypertexnames=false}
\else
	\usepackage[dvipdfm,%
					pdftitle={\TITLE},
					pdfauthor={\VG, \ZJ, \SV, \HZ},
					pdfsubject={\TITLE},
					pdfkeywords={\KEYWORDS},
					bookmarks=true,%
					pdfpagelabels=false,%
					colorlinks=true,%
					linkcolor=blue,%
					citecolor=blue]{hyperref}
\fi

\begin{document}
	
  \newfont{\mycrnotice}{ptmr8t at 7pt}
  \newfont{\myconfname}{ptmri8t at 7pt}
  \let\crnotice\mycrnotice%
  \let\confname\myconfname%
  
  \permission{Permission to make digital or hard copies of all or part of this work for personal or classroom use is granted without fee provided that copies are not made or distributed for profit or commercial advantage and that copies bear this notice and the full citation on the first page. Copyrights for components of this work owned by others than ACM must be honored. Abstracting with credit is permitted. To copy otherwise, or republish, to post on servers or to redistribute to lists, requires prior specific permission and/or a fee. Request permissions from Permissions@acm.org.}
  \conferenceinfo{DEBS'15,} {June 29 -- July 3, 2015, OSLO, Norway.}
  \copyrightetc{Copyright 2015 ACM \the\acmcopyr}
  \crdata{978-1-4503-3286-6/15/06...\$15.00.\\ http://dx.doi.org/10.1145/2675743.2772598}

\title{\TITLE}

\numberofauthors{4}
\author{
\alignauthor	\VG\\
					\VGADDR{\VGEMAIL}
\alignauthor	\ZJ\\
					\ZJADDR{\ZJEMAIL}
\alignauthor	\SV\\
					\SVADDR{\SVEMAIL}
\and
\alignauthor	\HZ\\
					\HZADDR{\HZEMAIL}
}

\date{\today}
\maketitle

\begin{abstract}
Pending...
\end{abstract}


\category{C.2.4}{Computer-Communication Networks}{Distributed Systems}[Distributed Applications]
\terms{Algorithms, Design}
\keywords{\KEYWORDS} 


%%%%%%%%%%%%%%%%%%%%%%%%%%%%%%%%%
%% Intro                       %%
%%%%%%%%%%%%%%%%%%%%%%%%%%%%%%%%%

\section{Introduction}
\label{sec:introduction}
The ACM DEBS 2016 Grand Challenge is the sixth in a series~\cite{jerzak2012debs, mutschler2013debs, jerzak2014debs, jerzak2015debs} of challenges which seek to provide a common ground and uniform evaluation criteria for a competition aimed at both research and industrial event-based systems. The goal of the 2016 DEBS Grand Challenge competition is to evaluate event-based systems for real-time analytics over high volume data streams in the context of graph models.

The underlying scenario addresses the analysis metrics for a dynamic (evolving) social-network graph. Specifically, the 2016 Grand Challenge targets following problems: (1) identification of the posts that currently trigger the most activity in the social network, and (2) identification of large communities that are currently involved in a topic. The corresponding queries require continuous analysis of a dynamic graph under the consideration of multiple streams that reflect updates to the graph.

The data for the DEBS 2016 Grand Challenge is based on the dataset provided together with the LDBC Social Network Benchmark~\cite{erling2015social}. DEBS 2016 Grand Challenge takes up the general scenario from the 2014 SIGMOD Programming Contest~\cite{DBLP:conf/sigmod/2014}, however, in contrasts to the SIGMOD contest, it explicitly focuses on processing streaming data and thus dynamic graphs. Details about the data, queries for the Grand Challenge, and information about evaluation are provided below.

%%%%%%%%%%%%%%%%%%%%%%%%%%%%%%%%%
%% Data                       %%
%%%%%%%%%%%%%%%%%%%%%%%%%%%%%%%%%
\section{Data}
\label{sec:data}
The data for the 2016 Grand Challenge is organized in four separate streams, each provided as a text file. The first input stream indicates when two users enter a "friendship" relationship -- see Table~\ref{table:friend} and Listing~\ref{code:friend}. The first input stream file name is $friendships.dat$.

\definecolor{lgray}{gray}{0.94}
\definecolor{llgray}{gray}{0.99}

\rowcolors{1}{lgray}{llgray}
\begin{table}[ht]
	\caption{The set of attributes used in the $friendships.dat$ input file}
	\centering 
	\begin{tabular}{r p{5.2cm}}
		\toprule
		Attribute		&	 Description\\
		\midrule
		ts			&	timestamp indicating when a friendship was established\\[2ex]
		user\_id\_1	&	id of one of the users\\[2ex]
		user\_id\_2	&	id of the other user\\[2ex]		
		\bottomrule 
	\end{tabular}
	\label{table:friend}
\end{table}



\lstset{}
\begin{lstlisting}[float=ht,caption={First line from the $friendships.dat$ file -- one attribute per line of listing},label={code:friend}]
XXX
YYY
ZZZ
PLEASE ADD ACTUAL DATA HERE
\end{lstlisting}	

The second input stream indicates when a users creates a new post -- see Table~\ref{table:post} and Listing~\ref{code:post}. The second input stream file name is $posts.dat$.

\rowcolors{1}{lgray}{llgray}
\begin{table}[ht]
	\caption{The set of attributes used in the $posts.dat$ input file}
	\centering 
	\begin{tabular}{r p{5.2cm}}
		\toprule
		Attribute		&	 Description\\
		\midrule
		ts			&	timestamp indicating when a post was created\\[2ex]
		post\_id	&	unique id of the post\\[2ex]
		user\_id	&	unique id of the user who created the post\\[2ex]		
		post		& 	string containing the post's content\\[2ex]		
		user		&   string containing the user name of the post creator\\[2ex]
		\bottomrule 
	\end{tabular}
	\label{table:post}
\end{table}



\lstset{}
\begin{lstlisting}[float=ht,caption={First line from the $posts.dat$ file -- one attribute per line of listing},label={code:post}]
XXX
YYY
ZZZ
PLEASE ADD ACTUAL DATA HERE
\end{lstlisting}

The third input stream indicates when a users commnets on a post -- see Table~\ref{table:comment} and Listing~\ref{code:comment}. The third input stream file name is $comments.dat$.

\rowcolors{1}{lgray}{llgray}
\begin{table}[ht]
	\caption{The set of attributes used in the $comments.dat$ input file}
	\centering 
	\begin{tabular}{r p{5.2cm}}
		\toprule
		Attribute		&	 Description\\
		\midrule
		ts			&	timestamp indicating when a comment was created\\[2ex]
		comment\_id	&	unique id of the comment\\[2ex]
		user\_id	&	unique id of the user who created the comment\\[2ex]		
		comment		& 	string containing the comment's content\\[2ex]		
		user		&   string containing the user name of the comment creator\\[2ex]
		comment\_replied		&   id of the comment being commented (-1 if this is a comment to a post)\\[2ex]
		post\_commented		&   id of the post being commented (-1 if this is a comment to a comment)\\[2ex]
		\bottomrule 
	\end{tabular}
	\label{table:comment}
\end{table}



\lstset{}
\begin{lstlisting}[float=ht,caption={First line from the $comments.dat$ file -- one attribute per line of listing},label={code:comment}]
XXX
YYY
ZZZ
PLEASE ADD ACTUAL DATA HERE
\end{lstlisting}

Each of the data files is sorted chronologically based on the timestamp ($ts$) attribute.


%%%%%%%%%%%%%%%%%%%%%%%%%%%%%%%%%
%% Query 1                     %%
%%%%%%%%%%%%%%%%%%%%%%%%%%%%%%%%%
\section{Frequent Routes Query}
The goal of the frequent routes query is to find the top ten most frequent routes during the last 30 minutes. A route is represented by a starting grid cell and an ending grid cell. All routes completed within the last 30 minutes are considered for the query. The output query results must be updated whenever any of the 10 most frequent routes changes. The output format for the result stream is shown in Listing~\ref{code:query1}.

\begin{lstlisting}[float=ht,caption={Output format for the frequent routes query},label={code:query1}]
pickup_datetime 
dropoff_datetime 
start_cell_id_1 
end_cell_id_1
...
start_cell_id_10 
end_cell_id_10 
delay
\end{lstlisting}

Where pickup\_datetime, dropoff\_datetime are the timestamps of the trip report that resulted in an update of the result stream, start\_cell\_id\_X the starting cell of the X$^{th}$-most frequent route, end\_cell\_id\_X the ending cell of the X$^{th}$-most frequent route. If less than 10 routes can be identified within the last 30 minutes, then NULL is to be output for all routes that lack data.

The attribute "delay" captures the time delay between reading the input event that triggered the output and the time when the output is produced. Participants must determine the delay using the current system time right after reading the input and right before writing the output. This attribute will be used in the evaluation of the submission.

The cells for this query are squares of 500 meters by 500 meters. The cell grid starts with cell 1.1, located at 41.474937, -74.913585 in Barryville. The coordinate 41.474937, -74.913585 marks the center of the first cell. Cell numbers increase towards the east and south, with the shift to east being the first and the shift to south the second component of the cell, i.e., cell 3.7 is 2 cells east and 6 cells south of cell 1.1. The overall grid expands 150km south and 150km east from cell 1.1 with the cell 300.300 being the last cell in the grid. All trips starting or ending outside this area are treated as outliers and must not be considered in the computation.


%%%%%%%%%%%%%%%%%%%%%%%%%%%%%%%%%
%% Query 1                     %%
%%%%%%%%%%%%%%%%%%%%%%%%%%%%%%%%%
\section{Profitable Areas Query}
The goal of profitable areas query is to identify areas that are currently most profitable for taxi drivers. The profitability of an area is determined by dividing the area profit by the number of empty taxis in that area within the last 15 minutes. The profit that originates from an area is computed by calculating the median fare including tip for trips that started in the area and ended within the last 15 minutes. The number of empty taxis in an area is the sum of taxis that had a drop-off location in that area less than 30 minutes ago and had no following pickup yet.

The result stream of the query must list the ten most profitable areas in the format presented in Listing~\ref{code:query2}.

\begin{lstlisting}[float=ht,caption={Output format for the profitable areas query},label={code:query2}]
pickup_datetime
dropoff_datetime
profitable_cell_id_1
empty_taxies_in_cell_id_1
median_profit_in_cell_id_1
profitability_of_cell_1
... 
profitable_cell_id_10
empty_taxies_in_cell_id_10
median_profit_in_cell_id_10
profitability_of_cell_10
delay
\end{lstlisting}

with attribute names containing cell\_id\_1 corresponding to the most profitable cell and attribute containing cell\_id\_10 corresponding to the 10$^{th}$ most profitable cell. If less than 10 cells were identified within the last 30 minutes, then NULL is to be returned for all cells that lack data. Query results must be updated whenever any of the 10 most profitable areas change. The pickup\_datetime and dropoff\_datetime in the output are the timestamps of the trip report that triggered the change.

The attribute "delay" captures the time delay between reading the input event that triggered the output and the time when the output is produced. Participants must determine the delay using the current system time right after reading the input and right before writing the output, i.e., including the serialization and deserialization time but excluding the disk IO time. 

Profitable areas query uses the same numbering scheme as for frequent routes query, however it uses a different resolution. In this query one should assume a cell size of 250m X 250m, i.e., the area to be considered spans from cell 1.1 to cell 600.600.

%%%%%%%%%%%%%%%%%%%%%%%%%%%%%%%%%
%% Additional Remarks          %%
%%%%%%%%%%%%%%%%%%%%%%%%%%%%%%%%%
\section{Additional Remarks}

%%%%%%%%%%%%%%%%%%%%%%%%%%%%%%%%%
%% License                     %%
%%%%%%%%%%%%%%%%%%%%%%%%%%%%%%%%%
\section{License}
All solutions submitted to the DEBS 2016 Grand Challenge are open source under the BSD license: \url{https://opensource.org/licenses/BSD-3-Clause}. A solution incorporates concepts, queries, and code developed for the purpose of solving the Grand Challenge. If a solution is developed within the context of, is built on top of, or is using an existing system or solution which is licensed under a different license than BSD, then such an existing solution or system maintains its existing license.

%%%%%%%%%%%%%%%%%%%%%%%%%%%%%%%%%
%% Acknowledgements            %%
%%%%%%%%%%%%%%%%%%%%%%%%%%%%%%%%%
\section{Acknowledgements}
The DEBS Grand Challenge Organizing Committee would like to explicitly thank WSO2 (\url{http://wso2.com}) for sponsoring the DEBS 2016 Grand Challenge prize and the LDBC Council (\url{http://www.ldbcouncil.org}) for their help in preparing the test data set.

%%%%%%%%%%%%%%%%%%%%%%%%%%%%%%%%%
%% References                  %%
%%%%%%%%%%%%%%%%%%%%%%%%%%%%%%%%%
\bibliographystyle{plain}
\bibliography{debsgc2016}

%\balancecolumns

\end{document}
